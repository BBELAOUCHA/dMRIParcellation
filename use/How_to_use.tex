\documentclass[a4paper,10pt]{article}
\usepackage[utf8]{inputenc}
\usepackage{listings}
\usepackage{color}
 
\definecolor{codegreen}{rgb}{0,0.6,0}
\definecolor{codegray}{rgb}{0.5,0.5,0.5}
\definecolor{codepurple}{rgb}{0.58,0,0.82}
\definecolor{backcolour}{rgb}{0.95,0.95,0.92}
 
\lstdefinestyle{mystyle}{
    backgroundcolor=\color{backcolour},   
    commentstyle=\color{codegreen},
    keywordstyle=\color{magenta},
    numberstyle=\tiny\color{codegray},
    stringstyle=\color{codepurple},
    basicstyle=\footnotesize,
    breakatwhitespace=false,         
    breaklines=true,                 
    captionpos=b,                    
    keepspaces=true,                 
    numbers=left,                    
    numbersep=5pt,                  
    showspaces=false,                
    showstringspaces=false,
    showtabs=false,                  
    tabsize=2
}
 
\lstset{style=mystyle}
 
%opening
\title{Cortical surface parcellation based on dMRI information and Mutual Nearest Neighbor condition}
\author{Brahim Belaoucha}

\begin{document}

\maketitle

This algorithm uses the dMRI information (tractogram of seeds) to divide the cortical surface into regions that have the highest Connectivity profiles (tractograms).

The algorithm takes the following inputs;


 \textbf{input\_path}: It is a matlab file that contain the followings:
\begin{itemize}
 \item Faces: The faces of the mesh (cortical surface) ($N\times 3$).
 \item Vertices: The coordinates of the vertices of the mesh (in anatomical or diffusion space) ($n\times 3$).
 \item Normal: The normal vector at each vertex ($n\times 3$).
 \item Mesh connectivity: matrix that locate the edges between the vertices ($n\times n$).
\end{itemize}

 \textbf{output\_path}: path to the folder where you want to save the data. Be sure you have the right to write in this path.

 \textbf{coordinate}: path to the file containing the coordinates ($n\times 3$ of int elements) of the mesh in the diffusion space. 

 \textbf{tracto\_path}: path to the tractograms. They must be in Nifti format ".nii.gz".

 \textbf{tracto\_prefix}: The beginning of the tractogram name. The tractogram obtained from FSL have the following name: tracto\_prefix\_x\_y\_z.nii.gz, where 
(x,y,z) is the coordinate of the seed.

 \textbf{Regions}: is a list of the required number of regions. Remember that this algorithm does not give the exact number of regions as required but it uses
this list to stop growing big regions.

 \textbf{Excluded\_seeds}: path to the file containing the seeds that will be excluded from the parcellation (ex.Thalamus, or one hemisphere).  

 \textbf{SM\_method}: the name of the similarity measure used in the parcellation. Fow now there is: "Correlation" (default), "Ruzicka", "Roberts", "Tanimoto", "Motyka", "Cosine"

 \textbf{Cvariance}: The coefficient of variance threshold is used to stop merging regions that have high variance of the similarity measure values. 

 
The algorithm's output is the Labels at each iteration in the path output\_path/RealLabel/ in '.vtk' format. Also the parcellation after merging small regions ($<0.1\frac{nbr_{seed}}{R}$) with big ones.
Few results concerning the mean and the std of the similarity measure values can be found in "results.txt".

\section{In a terminal (or Cluster)}
It is easy and straight forward: 

\noindent \textbf{python run\_parcellation} \textbf{-i} input\_path \textbf{-o} output\_path \textbf{-seed} coordinate \textbf{-t} tracto\_path \textbf{-tb} tracto\_prefix \textbf{-NR} Regions \textbf{-Ex} Excluded\_seeds \textbf{-sm} SM\_method -\textbf{cvth} Cvariance

The similarity method and number of regions should be in the following format; 
\begin{itemize}
 \item SM\_method: "Cosine,Ruzicka,Tanimoto,Motyka,Roberts"
 \item Regions: "1000,500,250"
\end{itemize}


\section{Run in python Code}
\begin{lstlisting}[language=Python]
import Region_preparation as RP 
from Cortical_surface_parcellation import Parcellation as CSP

Vertices,Connectivity,Faces,Normal,coordinate=[],[],[],[],[]
# Different similarity measures
SM_method=["Cosine","Ruzicka","Tanimoto","Motyka","Roberts"]
# R values
Regions=[1000,500,250]
# Initialize the parcellation
Parcel=CSP(tracto_path,tracto_prefix, output_path)
# The cortical surface that will be visualize in vtk
Mesh_plot=RP.Mesh(Vertices,Connectivity,Faces,Normal)
# Exclude the seeds
Excluded=[]
# Run the parcellation
Parcel.Parcellation_agg(coordinate,Connectivity,Excluded,
                        Regions,SM_method,Mesh_plot,Cvariance)
\end{lstlisting}

\noindent Both executions will create folders at the location output\_path with the name of the similarity measures (ex. output\_path/Cosine/500/).

\end{document}
